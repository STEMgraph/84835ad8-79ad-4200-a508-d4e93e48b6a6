\learningobjective{At the end of this challenge, the scholar will be able to use Git to manage a local repository effectively, understand the roles of the working tree, staging area, and commit history, navigate through commits using git log, and practice essential operations such as checking out commits, reverting to the latest changes, and discarding modifications.}
\begin{challenge}
    \chatitle{Introduction to Git: Managing a Local Repository}
    \begin{chadescription}
    Git is a tool, that was developed by Linus Torvalds, to manage source code in distributed software projects.
    There are a few challenges, when it comes to working with multiple developers on the same project.
    Git is one possible way that aims to mitigate these challenges and provide developers with a useful tool for keeping track of changes and collaborating on their codebase.
    It is often times called \textit{version control} or \textit{source control}.
    Anyway, that is a description of its use, not what it is: a tool for managing collections of files within a directory.
    Understanding the complete functionality of Git is beyond the scope of this challenge and takes multiple years of study, as well as constant practice.
    That being said, we will focus in this challenge on the basics of working with a local Git repository.
    As already mentioned, Git helps you keeping track of changes in a specific directory.
    In order to do that, Git needs its own storage space, sort of like an archive of all changes. 
    This storage space is called a \textit{repository} and in comparison to other database-management systems, it is not global to your computer, but rather local to each individual directory, that is managed by Git.
    Git does this, by creating a directory called \texttt{.git} in the root of your project and storing all changes in this directory.
    Don't worry too much: you won't have to deal with this directory directly, but rather through a handful of commands, that will be introduced in this challenge.
    You will understand key concepts like the working tree, staging area, and commit history, and practice navigating and managing changes in your repository.
    By the end of this challenge, you will be able to create a local repository, make and review commits, navigate through commit history, and discard unwanted changes confidently.

    \textbf{Note:} Before you start, make sure, that you have Git installed on your computer.
    Run \texttt{git --version} to check if it is installed and if it is, then you can continue with the challenge.
    Otherwise you can use your favourite package manager to install it.
    For example on Ubuntu: \texttt{sudo apt install git}.
    \end{chadescription}

    \begin{task}
        In order for us to work correctly with Git, lets configure it to a certain state. 
        We do that by running the following commands:
        \begin{lstlisting}
        git config --global user.name "Your Name"
        git config --global user.email "Your Email"
        git config --global core.editor "vim"
        git config --global init.defaultBranch master
        \end{lstlisting}
    \end{task}

    \begin{task}
        Since Git isn't working globally, each directory that we want to keep an eye on needs to be initialized as a repository manually using \texttt{git init}.
        This command will create a new directory, called \texttt{.git}, in the current directory and initialize the repository there.
        Your repository is empty at this point, but it's ready to receive changes.
        Changes are done and saved in the repository in a multi step process.
        The most important part of this process is, that you are always aware of the current state of your repository.
        You can figure that out, by using \texttt{git status}.
        We start by changing the contents of our directory, we tell Git which exact changes we want to save to our repository, and finally we store - or commit - the changes to our repository.
        When we are doing our commit we will be asked to decorate our commit with a message.
        The terminal will open a text editor for us to write our message.
        
        \begin{enumerate}
            \item Create a new directory named \texttt{my-git-project} in your home directory.
            \item Navigate into the directory and run \texttt{git status}.
            \item As you have seen from the output of \texttt{git status}, the directory is empty. We need to initialize it as a Git repository using \texttt{git init}. This creates the hidden directory \texttt{.git} in the current directory. Run \texttt{git status} again to see the changes.
            \item Create a new file named \texttt{file.txt} and add some text to it. Run \texttt{git status} and see has changed.
            \item Mark the file for the next commit using \texttt{git add file.txt}. This is also refered to as \texttt{staging}. You guessed it: run \texttt{git status} to see the changes, again. 
            \item Commit (or store) the changes by calling \texttt{git commit}. This creates a savepoint of all of the files, you added to the staging area. Next step? Run \texttt{git status}.
            \item \texttt{vim} (or your preferred text editor) will open. Use it to formulate a one-line commit message saying \texttt{"Initial commit"}. Actually, this is a pretty common message for the first commit of a project. We will investigate more sophisticated commit messages later.
            \item After modifying your textbuffer, save the file and close \texttt{vim} by pressing \texttt{Esc} and \texttt{:wq} and \texttt{Enter}.
        \end{enumerate}
        You can run \texttt{git status} over and over again. 
        It will not change anything in your repo, but will help you to understand the state of your repository.
        A common excercise is to run \texttt{git status} after each commit but try to guess correctly what the output will be.

        \begin{questions}
            \item What does the command \texttt{git init} do?
            \item How can you check which files have been staged for the next commit?
            \item What is the purpose of the commit message, and why is it important?
        \end{questions}
    \end{task}

    \begin{task}
        The Git repository is managed in three areas, that all exist simultaneously:
        \begin{itemize}
            \item \textbf{Working Tree}: The files you're actively editing.
            \item \textbf{Staging Area}: Files marked to be included in the next commit.
            \item \textbf{Repository}: The committed changes stored in Git.
        \end{itemize}
        Let's explore how these areas work together:
        \begin{enumerate}
            \item Modify \texttt{file.txt} by adding a new line of text but do not stage the changes.
            \item Run \texttt{git status} and note the message about the modified file.
            \item Stage the changes using \texttt{git add file.txt}. Run \texttt{git status} again to observe how the status changes.
            \item Commit the staged changes with the message \texttt{"Added more content to file.txt"}.
            \item Modify \texttt{file.txt} again, but this time, stage it with \texttt{git add file.txt}, and then directly commit using \texttt{git commit -m "Quick commit"}. Observe how the commit message is provided inline.
        \end{enumerate}

        \begin{questions}
            \item What does \texttt{git add} do, and how does it move changes from the working tree to the staging area?
            \item Why is it useful to have a staging area separate from the working tree?
            \item How does providing a commit message inline with \texttt{-m} differ from the default method of opening a text editor?
        \end{questions}
    \end{task}

    \begin{task}
        In Git, every commit is stored with a unique identifier, making it possible to navigate through the repository's history.
        The identifier is calculated from all the changes that have been made to the repository since the last commit.
        The method of calculating identifiers like this is called a \textbf{hashing}.
        Its result is a 40 character long string, that is unique for each commit and is called a \textbf{commit hash}.
        \begin{enumerate}
            \item Use \texttt{git log --oneline} to view a compact summary of the commit history.
            \item Identify the hash of the initial commit and check it out using \texttt{git checkout <hash>}.
            \item Run \texttt{git status} and observe that you are now in a detached HEAD state.
            \item Open \texttt{file.txt} and compare its content to the latest version you remember editing.
            \item Return to the latest commit by running \texttt{git checkout master}.
        \end{enumerate}

        \begin{questions}
            \item What information does \texttt{git log --oneline} provide compared to \texttt{git log}?
            \item What does the detached HEAD state mean, and why is it important to return to the master branch after checking out an old commit?
            \item How does Git ensure the safety of your current changes when navigating through the history?
        \end{questions}
    \end{task}

    \begin{task}
        Discarding unwanted changes is a critical skill when working with Git.
        Sometimes we end up making changes to the repository that we don't want to commit.
        We'd like to get rid of these changes, but we don't want to lose them.
        \begin{enumerate}
            \item Modify \texttt{file.txt} but do not stage the changes.
            \item Run \texttt{git status} to confirm that the file has been modified.
            \item Discard the changes using \texttt{git restore file.txt}. Open \texttt{file.txt} to verify the changes are gone.
            \item Make another modification, stage it using \texttt{git add file.txt}, and then unstage it using \texttt{git restore --staged file.txt}. Verify the file's state with \texttt{git status}.
            \item Modify \texttt{file.txt} again, but this time, discard all changes in the working directory with \texttt{git restore .}.
        \end{enumerate}

        \begin{questions}
            \item How does \texttt{git restore} differ from \texttt{git checkout} when used to discard changes?
            \item Why is it important to review your changes carefully with \texttt{git status} before discarding them?
            \item What is the difference between restoring changes in the working tree and unstaging changes from the staging area?
        \end{questions}
    \end{task}


    \begin{advice}
        Git is a powerful tool, but it can seem complex at first. 
        Focus on understanding the basic workflow: modifying files in the working tree, staging changes, and committing them to the repository. 
        Commands like \texttt{git log}, \texttt{git checkout}, and \texttt{git restore} are essential for navigating and managing your repository. 
        Always review your changes carefully using \texttt{git status} before committing or discarding them. 
        With practice, you will gain confidence and develop efficient habits for managing code with Git.
    \end{advice}
\end{challenge}
